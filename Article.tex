\documentclass[a4paper, 12pt]{article}

\usepackage[UTF-8]{ctex}
\usepackage{indentfirst}
\usepackage{amsmath}

\setlength{\parindent}{2em}
\numberwithin{equation}{section}

\begin{document}

    \title{''双碳''目标下低碳建筑研究}
    \author{}
    \date{}
    \maketitle

    \centerline{\textbf{\LARGE{摘要}}}

    \textbf{\Large{关键词}}

    {\centering{\section{问题重述}}}
        \subsection{问题背景}
        “双碳”即碳达峰与碳中和的简称,我国力争2030年前实现碳达峰,2060年前实现碳中和。
        “双碳”战略倡导绿色、环保、低碳的生活方式。我国加快降低碳排放步伐,大力推进绿色低碳科技创新,以提高产业和经济的全球竞争力。
        低碳建筑是指在建筑材料与设备制造、施工建造和建筑物使用的整个生命周期内,减少化石能源的使用,提高能效,降低二氧化碳排放量。

        \subsection{目标任务}
            \textbf{问题一:}现在有一间长 $ 4 m $ 、宽 $ 3 m $ 、高 $ 3 m $ 的单层平顶单体建筑,墙体为砖混结构,厚 $ 30 cm $ ( 热导系数 $ 0.3 W / m^{2} \cdot K $ ),
            屋顶钢筋混凝土浇筑,厚度 $ 30 cm $ ( 热导系数 $ 0.2 W / m^{2} \cdot K $ ) ,
            门窗总面积 $ 5 m^{2} $ ( 热导系数 $ 1.6 W / m^{2} \cdot K $ ),
            地面为混凝土 (热导系数 $ 0.25 W / m^{2} \cdot K $ ),
            该建筑所处地理位置一年(按365天计算)的月平均温度(单位:$ ^{\circ} C $ )见下页表。 \\
            \begin{table}
                \centering
                \begin{tabular}{|l|c|c|c|c|c|c|c|c|c|c|c|c|} \hline
                    月份 & 1 & 2 & 3 & 4 & 5 & 6 & 7 & 8 & 9 & 10 & 11 & 12 \\ \hline
                    平均温度 & -1 & 2 & 6 & 12 & 22 & 28 & 31 & 32 & 26 & 23 & 15 & 2\\ \hline
                \end{tabular}
            \end{table}
            假设该建筑物内温度需要一直保持在 $ 18 - 26 ^{\circ}C $ ,在温度不适宜的时候要通过电来调节温度,消耗 $ 1 kW \cdot h $ 电相当于 $ 0.28kg $ 碳排放。
            请计算该建筑物通过空调(假设空调制热性能系数\textit{COP}为3.5,制冷性能系数\textit{EER}为2.7)调节温度的年碳排放量。
            (尽量使用本题所给条件计算碳排放,不考虑其他损耗)

            \textbf{问题二:}在居住建筑的整个生命周期 (建造、运行、拆除)中,影响碳排放的因素有很多,
            如建筑设计标准、气候、建材生产运输、地区差异、建造拆除能耗、装修风格、使用能耗、建筑类型等。
            请查找、分析资料,建立数学模型,找出与上述因素相关度大且易于量化的指标,
            基于这些指标对居住建筑整个生命周期的碳排放进行综合评价。

            \textbf{问题三:}在问题2的基础上,分别考虑建筑生命周期三个阶段的碳排放问题,
            查找相关资料,建立数学模型,对2021年江苏省13个地级市的居住建筑碳排放进行综合评价,
            并对所建评价模型的有效性进行验证。

            \textbf{问题四:}准确的碳排放预测能够为制定减排政策、优化低碳建筑设计提供重要的参考依据。
            建立碳排放预测模型,基于江苏省建筑全过程碳排放的历史数据,对2023年江苏省建筑全过程的碳排放量进行预测。

            \textbf{问题五:}请结合前面的讨论给出江苏省建筑碳减排的政策建议。


    {\centering{\section{问题分析}}}
        \subsection{问题一}
            问题一要求计算通过空题调节温度产生的年碳排放量。
            我们需先求出空调制热和制冷分别消耗的电量,借此求出空调消耗的电量,最后转换成碳排放。

            首先计算出建筑物各个月的能量需求量。设室内温度要维持的温度为 $ t_{in} $, 室外温度为 $ t_{out} $,
            当月该地区平均温度为 $ t_{ave} $ ,方便起见,我们规定
            \begin{equation*}
                t_{out} = t_{ave}
            \end{equation*}
            \begin{equation*}
                t_{in} =
                \begin{cases}
                    18 ^{\circ}C & \text{ $ t_{out} < 18 ^{\circ}C $ } \\
                    t_{out} & \text{ $ t_{out} \in [18 ^{\circ}C, 26 ^{\circ}C] $ } \\
                    26 ^{\circ}C & \text{ $ t_{out} > 26 ^{\circ}C $}
                \end{cases}
            \end{equation*}
            我们使用单位面积热负荷计算每个月的能量需求量。单位面积热负荷可由以下公式计算
            \begin{equation}
                q = U \cdot A \cdot |\Delta t|
            \end{equation}
            其中,\textit{q}为单位面积热负荷,\textit{U}为总传热系数,\textit{A}为总传热面积, $ \Delta t = t_{in} - t_{out} $ 。\\
            总传热系数\textit{U}可由以下公式计算
            \begin{equation}
                U = \frac{1}{R_{wall} + R_{roof} + R_{air}}
            \end{equation}
            其中 $ R_{wall} $, $ R_{roof} $, $ R_{air} $ 分别为墙体、屋顶、空气层的热阻。 \\
            由于空气层的热阻很小,在此忽略不计。墙体和屋顶的热阻可以由以下公式计算
            \begin{equation}
                R = \frac{d}{\lambda}
            \end{equation}
            其中\textit{d}为墙体或房顶的厚度, $ \lambda $ 为其热传导率。

            计算得到总传热系数\textit{U}后即可进一步计算得出建筑物每月单位时间的能量需求量 $ Q_{heat} = q \cdot c $,其中\textit{c}为该月份所含单位时间数目,
            再根据 $ \Delta t $ 判断制热还是制冷,进而计算得出需电量 $ Q_{elec} $ 。 \\
            需电量的计算函数为
            \begin{equation}
                Q_{elec} =
                \begin{cases}
                    \frac{Q_{heat}}{EER} & \text{ $ \Delta t < 0 $ } \\
                    0 & \text{ $ \Delta t = 0 $ } \\
                    \frac{Q_{heat}}{COP} & \text{ $ \Delta t > 0 $ }
                \end{cases}
            \end{equation}

            最后根据需电量与碳排放的换算关系 $ m = Q_{elec} \cdot 0.28 $ 求出每月碳排放后累加,即得到年度碳排放量。

\end{document}
