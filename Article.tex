\documentclass[a4paper, 12pt]{article}

\usepackage[UTF-8]{ctex}
\usepackage{indentfirst}
\usepackage{amsmath}
\usepackage{tabularray}
\usepackage{xcolor}
\usepackage{listings}
\usepackage{subfigure}
\usepackage[graphicx]{realboxes}
\usepackage{diagbox}
\usepackage{multirow}

\lstset{
    basicstyle          =   \sffamily,
    keywordstyle        =   \bfseries,
    commentstyle        =   \rmfamily\itshape,
    stringstyle         =   \ttfamily,
    flexiblecolumns,
    numbers             =   left,
    showspaces          =   false,
    numberstyle         =   \zihao{-5}\ttfamily,
    showstringspaces    =   false,
    captionpos          =   t,
    frame               =   lrtb,
}

\lstdefinestyle{Python}{
    language        =   Python,
    basicstyle      =   \zihao{-5}\ttfamily,
    numberstyle     =   \zihao{-5}\ttfamily,
    keywordstyle    =   \color{blue},
    keywordstyle    =   [2] \color{teal},
    stringstyle     =   \color{magenta},
    commentstyle    =   \color{red}\ttfamily,
    breaklines      =   true,
    columns         =   fixed,
    basewidth       =   0.5em,
}

\setlength{\parindent}{2em}
\numberwithin{equation}{section}

\begin{document}

    \title{基于''XXX''的低碳建筑研究}
    \author{}
    \date{}
    \maketitle

    \centerline{\textbf{\LARGE{摘要}}}

    \textbf{\large{关键词}}

    {\centering{\section{问题重述}}}
        \subsection{问题背景}
        “双碳”即碳达峰与碳中和的简称,我国力争2030年前实现碳达峰,2060年前实现碳中和。
        “双碳”战略倡导绿色、环保、低碳的生活方式。我国加快降低碳排放步伐,大力推进绿色低碳科技创新,以提高产业和经济的全球竞争力。
        低碳建筑是指在建筑材料与设备制造、施工建造和建筑物使用的整个生命周期内,减少化石能源的使用,提高能效,降低二氧化碳排放量。

        \subsection{目标任务}
            \textbf{问题一:}计算给定建筑通过空调调节温度的年碳排放量。

            \textbf{问题二:}建立综合评价模型,找出易于量化的指标,评估居住建筑整个生命周期的碳排放。

            \textbf{问题三:}基于问题二,考虑建筑生命周期三个阶段的碳排放问题,对江苏省13个地级市的居住建筑进行评价,验证模型的有效性。

            \textbf{问题四:}建立碳排放预测模型,基于江苏省建筑全过程碳排放的历史数据,对2023年江苏省建筑全过程的碳排放量进行预测。

            \textbf{问题五:}结合前面的讨论给出江苏省建筑碳减排的政策建议。


    {\centering{\section{问题分析}}}
        \subsection{问题一}
            问题一要求计算通过空题调节温度产生的年碳排放量。
            我们需先求出空调制热和制冷的热量,借此通过\textit{COP}和\textit{EER}求出空调消耗的电量,最后转换成碳排放。
            其中\textit{COP}和\textit{EER}的定义分别为
            \begin{equation}
                COP = \frac{Q_{heat}}{W},\hspace{2em} EER = \frac{Q_{cold}}{W}
            \end{equation}
            $ Q_{heat} / Q_{cold} $指的是单位时间内的制热/制冷量,单位为\textit{J},
            公式中\textit{W}指的是单位为时间内空调消耗的功率,单位为\textit{W}

            首先计算出建筑物各个月的能量需求量。设室内温度要维持的温度为 $ t_{in} $,室外温度为 $ t_{out} $,
            当月该地区平均温度为 $ t_{ave} $ ,方便起见,我们规定
            \begin{equation*}
                t_{out} = t_{ave}
            \end{equation*}
            \begin{equation*}
                t_{in} =
                \begin{cases}
                    18 ^{\circ}C & \text{ $ t_{out} < 18 ^{\circ}C $ } \\
                    t_{out} & \text{ $ t_{out} \in [18 ^{\circ}C, 26 ^{\circ}C] $ } \\
                    26 ^{\circ}C & \text{ $ t_{out} > 26 ^{\circ}C $}
                \end{cases}
            \end{equation*}

            我们使用热传导方程计算用来需要制热/制冷的热量,其形式为
            \begin{equation}
                \Phi = \frac{\lambda \cdot A \cdot |\Delta T|}{\delta}
            \end{equation}
            其中$ \Phi $表示传热速率,$ \lambda $为导热系数,\textit{A}为传热面积,
            $ \Delta T $是室内外温度差,即$ t_{in} - t_{out} $,$ \delta $表示材料厚度。

            将建筑分成墙、门窗、房顶、地面四个部分,分别计算并累加即可得到需要制热/制冷的热量,设为$ Q_{make} $,
            由\textit{COP}和\textit{EER}的定义可得到需电量$ Q_{elec} $和热量$ Q_{make} $的转化关系
            \begin{equation}
                Q_{elec} =
                \begin{cases}
                    \frac{Q_{make}}{EER} & \text{ $ \Delta t < 0 $ } \\
                    0 & \text{ $ \Delta t = 0 $ } \\
                    \frac{Q_{make}}{COP} & \text{ $ \Delta t > 0 $ }
                \end{cases}
            \end{equation}

            最后根据需电量与碳排放的换算关系 $ m = Q_{elec} \cdot 0.28 $ 求出每月碳排放后累加,即得到年度碳排放量。


    {\centering{\section{模型假设}}}


    {\centering{\section{符号说明}}}


    {\centering{\section{模型的建立与求解}}}
        \subsection{问题一的模型建立与求解}

        \subsection{问题二的模型建立与求解}
            \subsubsection{建立层次结构模型}
                \begin{figure}[h]
                    \centering
                    \includegraphics[height=4.5cm,width=9.5cm]{层次分析法框架.png}
                    \caption{层次分析法框架}
                \end{figure}
                准则层中准则因素之间相互独立。 \\
                我们选择的准则因素有:生活使用能耗、地区差异、周边产业、建造与拆除能耗、生产运输。

            \subsubsection{构建成对比较矩阵及归一化}
                \begin{enumerate}
                    \item 构建比较矩阵 \\
                        \qquad 构造比较矩阵是通过比较同一层次上的各因素对上–层相关因素的影响作用.而不是把所有因素放在一起比较,即将同一层的各因素进行两两对比。
                        设某层有n个因素,$ x = \{x_{1}, x_{2} \dots x_{n}\} $要比较它们对上一层某一准则 (或目标)的影响程度,确定在该层中相对于某一准则所占的比重。
                        上述比较是两两因素之间进行的比较,比较时常取1~9尺度。

                        \begin{table}[h]
                            \centering
                            \begin{tabular}{|c|c|} \hline
                                尺度 & 含义 \\ \hline
                                1 & 第i个因素与第j个因素影响相同 \\ \hline
                                3 & 第i个因素与第j个因素影响稍强 \\ \hline
                                5 & 第i个因素与第j个因素影响较强 \\ \hline
                                7 & 第i个因素与第j个因素影响明显强 \\ \hline
                                9 & 第i个因素与第j个因素影响极端强 \\ \hline
                                2, 4, 6, 8 & 两相邻判断的中间值 \\ \hline
                            \end{tabular}
                        \end{table}

                        用$ a_{ij} $表示第i个因素相对于第j个因素的比较结果,则
                        \[ a_{ij} = \frac{1}{a_{ji}} \]
                        \begin{equation}
                            A = (a_{ij})_{n \times n} =
                            \begin{pmatrix}
                                a_{11} & a_{12} & \cdots & a_{1n} \\
                                a_{21} & a_{22} & \cdots & a_{2n} \\
                                \cdots & \cdots & \cdots & \cdots \\
                                a_{n1} & a_{n2} & \cdots & a_{nn}
                            \end{pmatrix}
                        \end{equation}

                        A则称为成对比较矩阵。

                        \newpage

                        \item 归一化 \\
                            对各城市的数据进行归一化处理
                            \begin{table}[h]
                                \centering
                                \begin{tabular}[h]{|l|c|c|c|c|c|} \hline
                                    \diagbox{指标}{数据}{城市} & 苏州 & 南京 & 南通 & 无锡 & 常州 \\ \hline
                                    直接 & 48 & 51.5 & 27 & 57.2 & 29 \\ \hline
                                    间接 & 126.3 & 134.3 & 118.7 & 97.8 & 71.6 \\ \hline
                                    运营 & 4.8 & 4.9 & 3.5 & 2.7 & 2.8 \\ \hline
                                    \multirow{3}{*}{归一化比例} & 0.226 & 0.242 & 0.127 & 0.269 & 0.126 \\ \cline{2-6}
                                    ~ & 0.230 & 0.245 & 0.216 & 0.178 & 0.130 \\ \cline{2-6}
                                    ~ & 0.257 & 0.262 & 0.187 & 0.144 & 0.150 \\ \hline
                                \end{tabular}
                            \end{table}
                    \end{enumerate}


                \subsubsection{层次单排序及一致性检验}
                    \begin{enumerate}
                        \item 层次单排序 \\
                            \textbf{和积法}(算术平均法):取判断矩阵\textit{n}个列向量归一化后的算术平均值,近似作为权重,即
                            \begin{equation}
                                W_{i} = \frac{1}{n} \sum_{j=1}^{n} \frac{a_{ij}}{\sum_{k=1}^{n} a_{kj}} (i = 1, 2, \cdots, n)
                            \end{equation}

                            \textbf{求根法}(几何平均法):将比较矩阵的各列(或行)向量求几何平均后归一化,可近似作权重,即
                            \begin{equation}
                                W_{i} = \sum_{j=1}^{n} \frac{(\prod_{j=1}^{n} a_{ij})_{\frac{1}{n}}}{\sum_{k=1}^{n} (\prod_{j=1}^{n})_{\frac{1}{n}}} (i = 1, 2, \cdots, n)
                            \end{equation}

                            \textbf{特征值法}:求出矩阵的最大特征值以及其对应的特征向量,对求出的特征向量进行归一化即可得到我们的权重。

                            在我们的模型中综合采用了三种方法得到权重,降低了单一方法带来的不确定性,使数据结果更加可靠。


                        \item 一致性检验 \\
                        通常情况下,由实际得到的判断矩阵不一定是一致的,即不一定满足传递性和一致性实际中,也不必要求一致性绝对成立,
                        但要求大体上是一致的,即不一致的程度应在容许的范围内主要考查以下指标:
                            \begin{enumerate}
                                \item 一致性指标\textit{CI}
                                    \begin{equation}
                                        CI = \frac{\lambda _{\max} - n}{n - 1}
                                    \end{equation}

                                \item 平均随机一致性指标\textit{RI} \\
                                    为衡量\textit{CI}的大小,引入随机一致性指标\textit{RI}:
                                    \begin{equation}
                                        RI = \frac{CI_{1} + CI_{2} + \cdots + CI_{n}}{n}
                                    \end{equation}
                                    其中,随机一致性指标\textit{RI}和判断矩阵的阶数有关,一般情况下,
                                    矩阵阶数越大,则出现一致性随机偏离的可能性也越大,对于阶数小于9,其对应关系如图:
                                    \begin{table}[h]
                                        \centering
                                        \begin{tabular}{l|c c c c c c c c c} \hline
                                            \textit{n} & 1 & 2 & 3 & 4 & 5 & 6 & 7 & 8 & 9 \\ \hline
                                            \textit{RI} & 0 & 0 & 0.58 & 0.90 & 1.12 & 1.24 & 1.32 & 1.41 & 1.45 \\ \hline
                                        \end{tabular}
                                    \end{table}

                                \item 检验系数\textit{CR} \\
                                    考虑到一致性的偏离有可能是由于随机原因造成的,因此在检验判断矩阵是否具有满意的一致性时,
                                    还需将\textit{CI}和\textit{RI}进行比较,得出检验系数\textit{CR},公式如下:
                                    \begin{equation}
                                        CR = \frac{CI}{RI}
                                    \end{equation}
                                    一般地,如果$ CR \le 0.1 $,则认为该判断矩阵通过一致性检验,$ A_{\max} $对应的特征向量\textit{W}可以作为排序的权重向量,此时
                                    \begin{equation}
                                        \lambda _{\max} \approx \sum_{i=1}^{n} \frac{(A \cdot W)_{i}}{nw_{i}} = \frac{1}{n} \sum_{i=1}^{n} \frac{\sum_{j=1}^{n} a_{ij}w_{j}}{w_{i}}
                                    \end{equation}
                                    否则就不具有满意一致性,需要调整对比较矩阵。
                            \end{enumerate}
                    \end{enumerate}


                \subsubsection{计算组合权重及得分}
                    得到最大特征值对应的特征向量
                    \[ T = [t_1 \quad t_2 \quad \cdots \quad t_n] \]
                    \newline
                    计算得到权重向量
                    \[ W = [w_1 \quad w_2 \quad \cdots \quad w_n ] \]
                    \begin{equation*}
                        w_i = \frac{t_i}{\sum_{i=1}^{n} t_i}
                    \end{equation*}
                    \newline
                    随后分别通过公式 (5.2)和公式 (5.3)求得两个权重向量, 最后将三个权重向量取算术平均作为最终的权重向量。
                    设指标评分矩阵为\textit{P},那么最后的得分矩阵\textit{S}为:
                    \[ S = P \cdot W \]


            \subsection{问题三的模型建立与求解}
                \subsubsection{模型的建立}
                    基于第二问所建模型,通过建筑生命周期的三个阶段确定代表性强的指标加入模型评价。我们选取的指标有: \\
                    \textit{建造阶段:建筑材料生产运输的碳排放系数、建造过程所产生的碳排放} \\
                    \textit{运行阶段:生活使用能耗、气候原因、地区差异、周边产业} \\
                    \textit{建造阶段:拆除过程的能耗}

                \subsubsection{模型的验证}
                    为了验证模型的有效性,我们将2021年江苏省13个地级市的居住建筑碳排放作为输入输入到所建立模型中进行综合评价,
                    并将得到的结果排名与实际的排名进行比较,结果如图:
                    \newpage
                    \begin{figure}[h]
                        \centering
                        \includegraphics[height=9cm,width=15cm]{contrast_q3.png}
                        \caption{CONTRAST}
                    \end{figure}
                    通过比较来看,评价模型的评价结果与实际结果吻合得较好,很好的验证了所建立模型的有效性。


            \subsection{问题四的模型建立与求解}
                \subsubsection{模型的分析}
                    本题是分析预测2023年江苏省建筑全过程的碳排放量。
                    由于要求基于江苏省建筑全过程碳排放的历史数据建立预测模型,用于分析江苏省未来建筑全过程的碳排放量,
                    因此我们采用信息不完全、不充分的预测系统——灰色预测,建立灰色预测模型GM (1,1)模型,基于历史时期数据去预测未来时期数据。

                \subsubsection{模型的建立}
                    \begin{enumerate}
                        \item \textbf{级比检验} \\
                            级比检验用于判断数据序列进行模型构建的适用性,如果通过了该检验,则可以使用灰色预测。
                            计算公式为:
                            \begin{equation}
                                \lambda (k) = \frac{x^{ (0)} (k - 1)}{x^{ (0)} (k)}, k = 2, 3, \ldots, n
                            \end{equation}
                            如果$ \lambda (k) $在区间
                            \[ (e^{-\frac{2}{n + 1}}, e^{\frac{2}{n + 2}}) \]
                            则说明可用GM (1,1)模型。
                    \end{enumerate}


    {\centering{\section{结果检验与误差分析}}}


    {\centering{\section{模型评价}}}


    {\centering{\section{模型推广与改进}}}


    {\centering{\section{参考文献}}}
    \newpage


    {\centering{\section{附录}}}
        \subsection*{附录A \hspace{2em} 问题一}
            \lstinputlisting[
                style       =   Python,
                caption     =   {\bf Question1},
                label       =   {Question1.py}
            ]{Question1.py}

        \subsection*{附录B \hspace{2em} 问题二}
            \lstinputlisting[
                style       =   Python,
                caption     =   {\bf Question2},
                label       =   {Question2.py}
            ]{Question2.py}

        \subsection*{附录C \hspace{2em} 问题三}
            \lstinputlisting[
                style       =   Python,
                caption     =   {\bf Question3},
                label       =   {Question3.py}
            ]{Question3.py}

        \subsection*{附录D \hspace{2em} 问题四}

        \subsection*{附录E \hspace{2em} 问题五}

\end{document}
